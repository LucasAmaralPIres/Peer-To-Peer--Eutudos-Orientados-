\documentclass[12pt]{article}

\usepackage{sbc-template}
\usepackage{graphicx,url}
\usepackage[utf8]{inputenc}
\usepackage[brazil]{babel}
\usepackage{verbatim}
\usepackage[final]{pdfpages}
     
\usepackage{xspace}
\newcommand{\FC} {Freechains\xspace}
\newcommand{\PtoP} {\emph{peer-to-peer}\xspace}

\sloppy

\title{Comparação de Sistemas de Reputação com o \FC}

\author{Lucas d'Amaral Pires\inst{1}} 

\address{Programa de Pós-Graduação em Engenharia \\ Universidade do Estado do Rio de Janeiro (UERJ) \\ Rio de Janeiro -- RJ -- Brasil}

\begin{document} 

\maketitle

\begin{resumo} 

É importante que usuários de redes \PtoP possam confiar uns aos outros para poder trocar conteúdo, sendo um dos grandes desafios desse tipo de rede garantir esse confiança entre \emph{peers}. O \FC é uma rede descentralizada \PtoP que promove a comunicação de diferentes usuários através de tópicos no qual estão inscritos. Para assegurar a confiabilidade dos usuários nesse tópico o \FC apresenta um sistema de reputação próprio. Nesse trabalho será avaliado o sistema de reputação do \FC em comparação com outros modelos de reputação para redes \PtoP propostos em outros artigos. Primeiro será introduzido brevemente cada sistema de reputação a ser comparado seguido da introdução do modelo de reputação do \FC. Depois o \FC será comparado diretamente com cada um dos sistemas mencionados anteriormente.
  
\end{resumo}


\section{Introdução} \label{sec:intro}

O mundo de hoje sofre constante mudanças decorrentes das novas tecnologias que surgem a todo momento. Uma das áreas que mudam e evoluem com mais frequência são as redes sociais, responsáveis pela maior disseminação de conteúdo na Internet.

Facebook, Twitter são exemplos de redes sociais centralizadas onde todos seu conteúdo esta armazenado em servidores que ficam a disponibilidade de seus usuários. O fato de serem sistemas centralizados trazem as vantagens de facilidade de gerência de conteúdo e usuários. Em outras palavras, eles conseguem localizar usuários maliciosos e seus conteúdos "malignos" com mais eficiência.

A grande quantidade de dados de seus usuários em um mesmo lugar facilita hackers a roubar essas informações. Além disso essas empresas podem vender esses dados para empresas de comerciais ou ainda usar esse dados para manipular o conteúdo visto pelos seus cliente de maneira que os mantenham mais tempo online usando suas plataformas.

Plataformas descentralizadas \PtoP por outro lado não mantém seus dados em um único lugar mas nos computadores pessoais de cada usuário que as utiliza. Isso não só dificulta hackers de roubar informações em grande quantidade mas também oferece maior privacidade aos seus usuários já que eles são totalmente responsáveis sobre quais informações desejam passar adiante ao invés de uma empresa que tem acesso a seus dados.

Ao contrário dos sistemas centralizados é mais difícil identificar usuários e conteúdos maliciosos já que cada usuário não contém um visão geral da rede mas somente de seus vizinhos. Isso possibilita  infiltração de usuários indesejados com mais facilidade. 

Sistemas de reputação são um tipo de solução implementada tanto em sistemas centralizados quanto sistemas descentralizados para validar usuários e conteúdos postados em um sistema. O objetivo de um sistema de reputação é encorajar usuários a postar conteúdos de qualidade e ser membros produtivos da rede. Dessa maneira também identificam usuários maliciosos e os restringem de maneira que não conseguem contaminar o resto do sistema.

Existem diversos tipos de sistema de reputação em utilização nos dias de hoje com a maior parte deles sendo voltada para reputação de usuários e maquinas com pouco sistemas focados no conteúdo em si. O \FC têm um diferencial pois propõem um sistema de reputação que avalia o conteúdo postado por cada usuário. 

Na Seção \ref{sec:trabrec} será analisado com mais detalhes os sistemas de reputação propostos nesses últimos anos. Na Seção \ref{sec:freechains} será apresentado o sistema de reputação do \FC. Na Seção \ref{sec:comparacao} o \FC será comparado com diversos sistemas de apresentados na Seção \ref{sec:trabrec}. Por fim na Seção \ref{sec:conclusao} a conclusão do trabalho é apresentada assim como ideias para trabalhos futuros.

\section{Trabalhos Relacionados} \label{sec:trabrec}

Ao longo dos anos vários sistemas de reputação foram propostos tanto para redes centralizadas como para as descentralizadas. A função principal da reputação é avaliar um usuário quanto a sua confiabilidade e integridade. A maior parte dos sistemas de reputação avaliam um usuário e seu comportamento na rede em que faz parte. 

Apesar de não ser aparente sistemas de reputação estão presentes na maior parte das aplicações dos dias hoje. O \emph{Facebook} apresenta o conceito de \emph{likes} onde usuários avaliam a postagem de outros usuários. \emph{Instagram}, \emph{Pinterest} e \emph{Twitter} também são exemplos de redes sociais que utilizam um sistema em que usuários avaliam uns aos outros. Os \emph{feedbacks} dos usuários ajudam com o crescimento ou a atenuação do escopo alcançado pelo publicador de conteúdo.  

O \emph{Stack Overflow} é um outro exemplo de um sistema de reputação onde usuários que postam e respondem mais perguntas recebem uma maior reputação no sistema, recebendo assim benefícios.

\subsection{Sistemas de Reputação Centralizados} \label{subsec:SRCentra}

Apesar de o \FC ser uma ferramenta descentralizada é importante avaliar também sistemas de reputação de sistemas centralizados. Como mencionando no início desta seção, sistemas de reputação estão mais evidentes no cotidiano das pessoas do que se imagina. As redes socais em geral são centralizadas e apresentam sistemas de reputação em que usuários se avaliam e assim são incentivados a postar mais conteúdo, aumentando o escopo da rede.

Nesse trabalho será apresentado dois sistemas de reputação centralizados sugeridos como melhoras para os atualmente utilizados pela \emph{Wikipedia} e \emph{Stack Overflow}. Esses sistemas foram escolhidos pelo fato que focam em reputação obtida através do conteúdo proposto por cada autor. Os sistemas de reputação descentralizados que serão avaliados nesse trabalho não apresentam essa característica, com a reputação não sendo avaliada pelo conteúdo postado mas sim pelas ações de cada usuário. Posteriormente na Seção \ref{sec:comparacao} essa característica de reputação pelo conteúdo que vai ser comparada com o \FC. 

Em seu trabalho \cite{adler2007content} apresenta um sistema de reputação para \emph{Wikipedia} com o intuito de melhorar a integridade das suas postagens. Ele propõem avaliar o usuário a partir do quanto a sua contribuição é mantida e o quanto é alterado em subsequentes edição daquela página. Em outras palavras, não é avaliado o quanto o usuário está ativo no sistema mas a qualidade de sua influência. 

Basicamente, pontos são fornecidos a usuários que realizam atualizações em verbetes que são mantidos por outros usuários e a soma desses pontos são a reputação do usuário. Atualmente a \emph{Wikipedia} não apresenta um sistema de reputação próprio e qualquer pessoa pode editar um verbete o que pode ser um problema em assuntos controversos com opiniões drasticamente diferentes.

O sistema proposto fornece a alternativa de que somente usuários com alta reputação poderiam editar alguns tópicos consideráveis sensíveis. Apesar disso o autor ressalta que o sistema não é próprio para avaliar autores de verbetes de situações cotidianas, já que essas situações estão em constante mudança e qualquer atualização da página contaria negativamente para o autor anterior.

O outro sistema centralizado visto foi proposto para o \emph{Stack Overflow}. Atualmente o \emph{Stack Overflow} apresenta um sistema de reputação que se baseia na atividade do usuário, ou seja, um usuário muito ativo no sistema ganha muita reputação. Em seu trabalho \cite{huna2016exploiting} apresenta o problema de usuários que usam o sistema simplesmente para ganhar reputação sem oferecer contribuições válidas.

Para combater esse problema o autor desenvolveu um sistema de reputação próprio para o \emph{Stack Overflow} avaliando o conteúdo das postagens ao invés da atividade do usuário. Dessa maneira um usuário que posta perguntas ou respostas consideradas relevantes ou importantes ganham mais reputação do que usuários que respondem simplesmente querendo ganhar reputação.

A solução apresentada não somente desincentiva usuários a postar uma resposta fraca ou ruim mas também eleva a posição de usuários que contribuem com conteúdo válido ou bom.

\subsection{Sistemas de Reputação Descentralizados} \label{subsec:SRDescen}

Em seu trabalho \cite{10.1145/1041680.1041681} faz um apanhado de tecnologias \PtoP de distribuição de conteúdo. O artigo apresenta um conjunto de definições importantes para os trabalhos desenvolvidos na área. Apesar de não focar em sistemas de reputação o trabalho foi essencial no desenvolvimento do conhecimento necessário para avaliar sistemas \PtoP e suas peculiaridades.

A seguir será analisado diversos sistemas de reputação propostos para redes descentralizadas \PtoP.

Em seus trabalhos \cite{dennis2015rep} e \cite{dennis2016rep} avaliaram e desenvolveram as ideias propostas em \cite{nakamoto2008peer} e propuseram uma sistema de reputação baseado em \emph{blockchain} com o intuito de promover um sistema que permite a participação entre usuários desconhecidos de maneira confiável. 

Esse sistema de reputação adquire várias características inerentes a uma \emph{blockchain} para garantir a confiabilidade da reputação dos usuários que participam da rede. Para isso é proposto a criação de uma \emph{blockchain} nova que tem o propósito de guardar a reputação obtida ao final de uma transação entre usuários.

Ao guardar as transações em uma \emph{blockchain} todos os usuários tem acesso ao histórico de reputação daquela rede e podem realizar consultas a reputação de outros usuários a qualquer momento. Por ser uma \emph{blockchain} o sistema de reputação se mostra confiável após algum tempo de serviço já que o poder computacional para um usuário malicioso "corromper" a rede cresce drasticamente.

Em seu artigo \cite{1231515} apresenta um sistema de confiança em redes \PtoP que variam entre usuários. Em seu sistema ele considera que o que é válido para um usuário pode não ser valido para outro e assim diferencia a reputação entre os usuários. Por exemplo, um usuários A prefere receber conteúdo de maneira rápida enquanto um usuário B prefere receber conteúdo de qualidade alta. Logo a reputação de de B para A é baixa e vice-versa. Em um caso de dois \emph{peers} com o mesmo interesse eles podem compartilhar informação de \emph{peers} com que também tem o mesmo interesse formando uma rede de confiança e alta reputação automaticamente entre membros parecidos.

Um sistema de reputação denominado \emph{PowerTrust} foi desenvolvido por \cite{zhou2007powertrust} com o intuito de se aproveitar do modelo \emph{power-law}. Em seus estudos eles observaram que alguns \emph{peers} tendem a congregar conteúdo e logo são mais ativos na rede. Assim a maior parte dos \emph{peers} se comunicam com esses super-usuários. 

Enquanto cada usuário mantém localmente a reputação de cada um de seus vizinhos de acordo com trocas de conteúdo já realizadas anteriormente, a reputação global de cada \emph{peer} tende a ser guardado nos super-usuários para fácil consulta. Esses super-usuários são dinamicamente escolhidos de acordo com a congregação de \emph{peers} neles e podem ser alterados de acordo com o comportamento da rede.

O mesmo autor também sugeriu um outro sistema de reputação  baseado no protocolo de \emph{gossip} em \cite{zhou2007gossip}. Nesse sistema de reputação denominado \emph{GossipTrust} o objetivo também é agregar as reputações de cada usuário conseguir um valor global de reputação. Essa agregação é feita através de \emph{gossip} entre os \emph{peers}.

Depois de um tempo aleatório cada \emph{peer} contata um de seus vizinhos para trocar informações de reputação com ele, a partir da informação recebida é calculado uma nova reputação dos usuários daquela rede. Cada \emph{peer} utiliza as informações recebidas para então realizar um algoritmo probabilístico com o intuito de chegar o mais perto possível do valor global real de reputação de cada \emph{peer}.

Em seu trabalho \cite{mortazavi2006cumulative} introduzem um sistema de reputação que busca justiça entre a colaboração de um usuário na rede a sua quantidade de reputação. Cada \emph{peer} tem uma variável que indica o nível de cooperação desejado que pode ser mudado por ele mesmo. Ao mesmo tempo ele tem um nível de reputação que só pode ser alterado por outros \emph{peers} que realizam troca de conteúdo com ele.

A quantidade de reputação que cada usuário possui influencia quais outros \emph{peers} ele tem acesso. Um usuário que coloca um alta nível de cooperação para si faz mais troca com outros usuários e tem sua reputação aumentada. Assim consegue receber um melhor serviço no futuro pois outros usuários de alta reputação querem trabalhar com ele já que ele tem uma boa reputação.

O último artigo a ser apresentado é o trabalho de \cite{gupta2003reputation}. Nesse artigo o autor apresenta dois  sistemas de reputação com base em crédito, ou seja, a reputação é uma unidade de troca de serviços.

No primeiro sistema denominado DCRC \emph{(Debit-Credit Reputation Computation)} a reputação de um usuário pode ser alterada da seguinte maneira: respondendo a pergunta de outros usuários quanto a conteúdo fornecido; fornecendo conteúdo para \emph{download}; baixando conteúdo de outros \emph{peers}; fornecer conteúdo raro (com pouca instâncias na rede). Nesse caso o usuários recebe reputação em todos os casos exceto baixar conteúdo, onde perde reputação para fazer tal. A reputação ganha não expira e pode ser usada a qualquer momento.

O segundo sistema é denominado CORC e nele não existe perda de crédito por baixar conteúdo de outros \emph{peers}, somente ganho de reputação, porém agora ela tem um limite temporal de expiração no qual se o usuário não usar a reputação ela é perdida. 

\section{Sistema de reputação do \FC} \label{sec:freechains}

O sistema \PtoP \FC é do tipo \emph{publish-subscribe} e oferece vários tópicos. Um usuário pode se inscrever em um tópico \emph{(subscribe)} e publicar \emph{(publish)} conteúdo. Os tópicos são diversos e podem ser sobre qualquer assunto sendo divididos 
em três categorias: \textbf{grupo privado, identidade pública e fóruns públicos}. 

Em \textbf{grupos privados} usuários podem se comunicar da mesma maneira que um grupo no \emph{WhatsApp}, assume-se que todos os integrantes do grupo são produtivos e confiáveis portanto todos os integrantes possuem reputação infinita e podem trocar conteúdo livremente.

Em tópicos de \textbf{identidade pública} somente um usuário pública mensagens e os outros usuários somente podem visualiza-las. Simular a uma notícia em um site de jornalismo onde o repórter escreve uma reportagem e os leitores não podem modifica-las. Nesse tipo de tópico o autor têm reputação infinita e pode postar o que quiser quantas vezes quiser.

Por fim, \textbf{fóruns públicos} são tópicos em que todos os seus assinantes podem publicar mensagens não tendo necessariamente confiança entre os usuários (o contrário dos grupos privados). Dessa maneira é necessário um sistema de reputação próprio e customizado para atender as necessidades de segurança e validade de usuários e conteúdos. O sistema de reputação implementado pelo \FC para fóruns públicos é detalhado a seguir.

\subsection{Fóruns Públicos} \label{subsec: forunspub}

O sistema de reputação criado tem o objetivo de evitar SPAM, notícias falsas e conteúdo ilícito. A reputação do \FC é voltada para avaliação de conteúdo, logo a reputação de cada usuário pode variar entre os diferentes tópicos em que esta escrito. 

A unidade \emph{rep} foi definida para controlar a quantidade de reputação que cada usuários têm em cada tópico. Ela pode ser transferida entre usuários através de \textbf{\emph{likes}} e \textbf{\emph{dislikes}} nas publicações de cada usuários.

Para facilitar a gerência da rede e evitar que usuários ganhem muita reputação e abusem dela foram impostos diversos limites de como unidades de \emph{rep} podem ser usadas. 

\subsubsection{Geração} \label{subsubsec:geracao}

Para poder haver qualquer tipo de validação de uma postagem no início do tópico é necessário que o primeiro usuário ao criar o tópico também ganhe reputação e assim consiga postar conteúdo. Logo, a primeira postagem de uma cadeia automaticamente adiciona trinta \emph{rep} ao autor. 

Postar conteúdo também gera \emph{rep}, dessa maneira usuários são recompensados por contribuir no tópico com reputação. Porém para os usuários não abusarem dessa regra e começarem a postar qualquer tipo de de conteúdo indiscriminadamente para conseguir reputação também foi estipulado que somente postagens com pelo menos 24 horas de vida contam com uma reputação positiva para o autor com no máximo de uma reputação por dia. 

Só existem duas maneiras de ganhar reputação em um tópico o que torna unidades de \emph{rep} valiosas, ao mesmo incentivando a produção de conteúdo, para ganhar mais reputação, mas também incentivando cautela no quanto de reputação deve ser gasto.  

\subsubsection{Consumo} \label{subsubsec:consumo}

Postar conteúdo com menos de 24 horas desde a última postagem consume uma \emph{rep} do usuário. Essa medida incentiva cada usuários a postar conteúdo de alta qualidade para conseguir \emph{likes} de outros usuários e compensar a reputação gasta.

Outra motivo importante dessa regra é que ela ajuda na prevenção de SPAM e outros ataques de usuários maliciosos. Já que para postar mensagens uma seguida da outra é consumido reputação do usuário ele não consegue encher o tópico de mensagens indesejadas já que sua própria reputação é rapidamente depletada. 

\subsubsection{Transferência} \label{subsubsec:transfe}

Existem duas maneira de transferir reputação entre diferentes usuários. A primeira é dando um \emph{like} em uma postagem de outros usuário. Nesse caso o usuário que postou uma mensagem ganha uma unidade de \emph{rep} e o usuário que deu o \emph{like} perde uma unidade de \emph{rep}. Em outras palavras, é realizado uma transferência de \emph{rep} de um usuário para outro. 

A vantagem de \emph{likes} para o tópico são o incentivo de conteúdo  bom ser postado na rede. Os diversos autores de conteúdo recebem uma vantagem em \emph{rep} para realizar postagens que agradem outros usuários inscritos naquele tópico.

A outra maneira que um usuário pode afetar outros usuários da rede são através de \emph{dislikes}. Nesse caso ao usuário dar um \emph{dislike} na postagem de outros usuário ele esta retirando uma unidade de \emph{rep} do autor da postagem ao mesmo que tempo que retira uma de suas próprias unidades de \emph{rep}. Caso uma postagem tenha pelo menos 5 \emph{dislikes} e o dobro de \emph{dislikes} comparado com \emph{likes} a postagem é banida da rede.  

A vantagem de \emph{dislike} é que ele permite a eliminação de conteúdo indesejado do tópico ao mesmo tempo que penaliza o usuário que fez aquela postagem. Isso incentiva usuários a postarem somente conteúdo relevante ao tópico para que não sejam penalizados. Por outro lado um \emph{dislike} retira \emph{reps} permanentemente da rede, o que é um incentivo ainda maior evitar conteúdo desagradável.

\subsubsection{Regras adicionais} \label{subsubsec:+regras}

Para evitar que um usuário ganhe muita reputação e futuramente abuse dessa reputação, existe um limite quanto a quantidade máxima de \emph{reps} que um usuário pode ter. Cada usuário pode ter no máximo trinta unidade de \emph{rep} a qualquer momento. Uma das vantagens dessa medida é que um usuário malicioso não consegue acumular reputação para futuramente lançar um ataque na rede. Os usuários também são incentivados a gastar sua reputação com \emph{likes} e \emph{dislikes} de maneira que não "percam" unidade de \emph{rep} ao atingir o limite e ficarem impossibilitadas de ganhar mais.

Outra medida que visa incentivar usuários a avaliarem o conteúdo da rede com \emph{likes} e \emph{dislikes} é o limite de 3 meses nos quais as unidades de \emph{rep} são válidas. Após 3 meses qualquer \emph{rep} não utilizado é deletado e perdido. 

Por fim, usuários sem reputação não podem propagar mensagens, ou seja, suas postagens ficam retidas e não são retransmitidas até que o usuário possa pagar a taxa de um \emph{rep}. Essa medida ajuda a manter a rede honesta e evita ataques de usuários maliciosos pois eles não conseguem mandar mensagens quando sua reputação e menor ou igual a zero.

\section{Comparações com \FC} \label{sec:comparacao}

Nesta Seção será comparado cada um dos sistemas de reputação apresentados na Seção\ref{sec:trabrec} com o sistema de reputação do \FC apresentado na Seção \ref{sec:freechains}. Os sistema de reputação serão comparados na ordem em que que foram apresentados na Seção \ref{sec:trabrec}.

\subsection{Comparação com Sistemas de Reputação Centralizados} \label{subsec:CompSRC}

Os próximos dois sistemas de reputação apresentados são feitos para redes centralizadas enquanto o \FC foi desenvolvido para um rede descentralizada, apesar disso a comparação é válida quando se observa que ambos os sistemas focam em reputação recebida pelo conteúdo postado pelos usuários e será nesse ponto que será focado a avaliação.

O primeiro sistema de reputação apresentado foi o de \cite{adler2007content} para a \emph{Wikipedia}. Uma das similaridades entre o \FC e o sistema proposto é o fato de que ambos os sistemas de reputação são influenciados pelo conteúdo postado por cada autor em suas respectivas plataformas. Dessa forma pode-se dizer que a reputação de um usuário é baseado em conteúdo.

Outra similaridade é o fato de que a unidade de reputação em \cite{adler2007content} incentiva os usuários a postarem conteúdos de qualidade a partir de recompensas recebidas por bom conteúdo.

Uma vantagem que o \FC apresenta é que é possível influenciar diretamente a reputação de outro usuário, permitindo que um usuários do \FC possa impedir que usuários malicioso contamine a rede enquanto postagens na \emph{Wikipedia} são anônimas, logo um usuário malicioso pode contaminar o conteúdo do verbete. O sistema de reputação não impede a ação de usuários maliciosos, ele apenas aumenta a confiabilidade no texto escrito por um usuários de alta reputação.

O outro sistema centralizado apresentado nesse trabalho foi o proposto por \cite{huna2016exploiting} para o \emph{Stack Overflow}.
Assim como o \FC essa proposta incentiva a usuários a postarem conteúdo de qualidade oferecendo reputação como recompensa. Em ambos os sistemas de reputação usuários podem votar no conteúdo postado, concedendo reputação para o autor da postagem ou tirando reputação de um usuário malicioso. 

Ambos os sistemas de reputação são bem similares oferecendo reputação por postar conteúdo e participação na rede, porém o \FC introduz um limite na quantidade máxima de reputação que um usuário pode e em custo para postar conteúdo com o intuito de evitar que um usuário malicioso acumule reputação e lance um ataque. O mesmo não é verdade no sistema proposto para o \emph{Stack Overflow}, a reputação de cada usuário é infinita e como não existe custo para postar conteúdo (somente o desencorajamento de perda de reputação) esse sistema é vulnerável a ataques de usuários maliciosos.

Um das diferenças que o \FC tem com os dois sistemas de reputação mencionados anteriormente é que um usuário pode ter diferentes reputações em diferentes tópicos. Não existe um valor universal de reputação de um usuário. Isso pode ser uma desvantagem para localizar um usuário malicioso que pode ser "banido" de um tópico mas ainda assim ser um membro da rede contaminando outros tópicos.  

\subsection{Comparação com Sistemas de Reputação Descentralizados} \label{subsec:CompSRD}

Os sistemas de reputação avaliados a seguir são baseados em redes descentralizadas assim como o \FC. Porém ao contrário do \FC  eles focam em ações de um usuário e não no conteúdo compartilhado por eles.

O sistema de reputação apresentado em \cite{dennis2015rep} e \cite{dennis2016rep} é baseado na tecnologia de \emph{blockchain}. A principal ideia do sistema deles é criar uma \emph{blockchain} responsável por guardar a reputação e as transações entre usuários. O \FC apresenta uma ideia similar, onde cada tópico apresenta uma cadeia no estilo de \emph{blockchain} e qualquer usuário pode verificar a cadeia para consultar a reputação de outros usuários.

Em termos de estrutura esse sistema de assemelha muito com o \FC na maneira de como guarda a reputação como na maneira como cresce. Ambos apresentam problemas quanto ao custo que guardar a reputação pode ter com o tempo (quantidade de bytes) e a demora de consulta em tópicos muito antigos.

Uma diferença que esse sistema com o \FC é o fato de serem guardados a reputação global de cada usuário, ou seja, todos as transações de um usuários são guardadas na \emph{blockchain} enquanto no \FC cada tópico tem sua própria reputação. Por um lado as cadeias no \FC crescem mais devagar e assim o peso delas demora para acumular quando comparado com o sistema proposto em \cite{dennis2015rep} e \cite{dennis2016rep}.

O trabalho realizado por \cite{1231515} descreve um sistema de reputação no qual a reputação de um usuário pode variar de acordo com as características procuradas. Usuários que buscam a mesma qualidade em seus \emph{peers} tendem a ficar juntos e formar uma sub-rede não dissimilar aos tópicos do \FC. 

O sistema de reputação de certa maneira funciona mais com o intuito de facilitar usuários a identificar outros usuários semelhantes na busca por conteúdo ao contrário do \FC onde a busca de usuários com interesses semelhante é anterior ao sistema de reputação mas durante a inscrição de tópicos.

Ao contrário do \FC, o sistema de reputação de \cite{1231515} não oferece nenhuma proteção contra usuários maliciosos, assumindo que todos os usuários que participam da rede têm boas intenções. O incentivo ao usuários participarem da troca de conteúdo de qualidade também não existe nesse rede, onde a consequência principal de seus sistema de reputação é a criação de sub-redes de usuários com os mesmo interesses.

Assim como o \FC o sistema proposto por \cite{zhou2007powertrust} tenta controlar o abuso de usuários maliciosos disponibilizando a reputação global de seus usuários para acesso comum. Porém enquanto o sistema proposto se utiliza de super-usuários o \FC se utiliza de mecanismos derivados da tecnologia \emph{blockchain} de maneira que todos os usuários tenham sempre a reputação de outros guardada. 

Uma falha inerente ao sistema que não está presente no \FC é o fato de que estão dependentes dos super-usuários que guardam reputações globais, caso eles todos fiquem \emph{offline} em um curto período de tempo a reputação global se perde e precisam ser selecionados novos super-usuários. No \FC qualquer usuário pode consultar seu registro de blocos para verificar a reputação de outro usuário, uma desvantagem é que essa abordagem do \FC ocupa mais espaço de disco.

O modelo \emph{GossipTrust} introduzido em \cite{zhou2007gossip} combina com o \FC já que ele apresenta a ideia de compartilhar reputação através de um algoritmo de \emph{gossip} da mesma maneira que os usuários do \FC propagam suas mensagens em um tópico. A vantagem que o \FC apresenta sobre o \emph{GossipTrust} é o fato de que a reputação exata de cada usuário pode ser consultada analisando a cadeia de mensagens enquanto no \emph{GossipTrust} as reputações são compartilhadas aleatoriamente com outros \emph{peers} e a reputação global é obtida através de um algoritmo probabilístico que demanda recursos de computação e não garante sucesso. 

O \FC busca incentivar seus usuários a compartilhar conteúdo de qualidade ao mesmo tempo que desencoraja usuários maliciosos de atuarem através de seu sistema de reputação. No trabalho realizado em \cite{mortazavi2006cumulative} o sistema de reputação deles busca uma balança entre a disposição de um usuário de participar na rede e sua reputação. Nesse sentido ambos os sistemas são similares pois buscam recompensar com reputação usuários participativos.

Apesar disso um efeito do sistema de \cite{mortazavi2006cumulative} é a aglomeração de usuários de alta reputação enquanto usuários novos ou de baixa reputação não conseguem alcança-los. O \FC e sua propagação de conteúdo estão acessíveis a todos os usuários de um tópico não importa o quanto tenham particionado até então.

O último artigo a ser comparado com o \FC, \cite{gupta2003reputation} apresenta dois sistema de reputação. O DCRC assim como o \FC apresenta o conceito de incentivar usuários a participar da rede oferecendo reputação como recompensa ao mesmo tempo que controla a quantidade de reputação retirando-a ao adquirir recursos de outros usuários. Apesar disso ele não apresenta um controle para evitar a ação de usuários maliciosos já que ter uma reputação baixa não impede eles de fornecerem arquivos contaminados.

O segundo sistema denominado CORC retira a perda de reputação o que o diferencia do \FC porém introduz a ideia de expiração de reputação após um tempo predeterminado o que é uma das medidas adotadas pelo \FC incentivar a participação de \emph{peers} com conteúdo.

Como mencionado anteriormente todos os sistemas de reputação para redes \PtoP descentralizados analisados tem foco em troca de arquivos e a reputação serve para identificar usuários confiáveis para a troca. Assim, foram comparados somente as características do \FC que podem gerar um paralelo com os sistemas propostos e os pontos em que são completamente diferentes e não faz sentido correlacionar os dois sistemas não foram mencionados.  

\section{Conclusão}\label{sec:conclusao}

Ainda para fazer

\bibliographystyle{sbc}
\bibliography{sbc-template}

\end{document}
